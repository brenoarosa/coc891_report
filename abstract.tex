O presente trabalho aborda a utilização de redes neurais convolucionais no processamento de linguagem natural.
Para tal, são mostradas as principais técnicas de pré-processamento e representação de texto presentes na literatura.
Por fim, se aborda o uso de redes convolucionais para classificação de sentimento de mensagens de Twitter usando supervisão distante,
não necessitando anotação manual dos dados, e uma comparação de seus resultados com técnicas de \textit{Machine Learning}
tradicionalmente aplicadas a texto.
