O trabalho de Go \textit{et at.}~\cite{go09} visa definir um sistema de classificação de sentimento sem a necessidade
de anotação manual dos dados para treinamento, tal característica se faz possível usando a técnica de supervisão
distante, como será explicada na seção \ref{distant_supervision}.
Apesar do termo \textit{análise de sentimento} ser ambíguo na literatura, nesse contexto estaremos tratando da extração
de polaridade, negativa ou positiva, de uma mensagem.
Em especial, as mensagens tratadas nesse trabalho são originárias do Twitter, serviço de microblogs.
Tais mensagens também são referidas como \textit{tweets}.

Embora já existisse uma ampla gama de estudos em análise se sentimento, Go \textit{et at.} apresentaram o primeiro
trabalho referente a \textit{tweets}.
A principal característica das mensagens de microblogs são seu tamanho reduzido, no caso do Twitter limitadas a
140 caracteres.
Devido a esse fato e ao meio no qual as mensagens estão inseridas, é observáveis variações do idioma especificas
deste contexto.
Outro traço notável é que por ser uma plataforma aberta, o Twitter é composto de mensagens dos mais diversos domínios.
Esse fator dificulta a tarefa a ser executada quando comparado com trabalhos de um domínio específico, como por exemplo
a análise de sentimento para avaliação de filmes a partir de comentários; ou na predição de flutuações do mercado
financeiro considerando artigos jornalísticos.
Além das características textuais, os \textit{tweets} também atributos como: republicação ou citação de mensagens, a
presença de fotos e video, as conexões entre os usuários, local de envio, horário, número de curtidas etc. Tais variáveis
podem colaborar na realização de tarefas como a análise de sentimento, porém, este trabalho abordará apenas as
características textuais.

A não dependência de anotação dos dados e a facilidade de coleta dos \textit{tweets}, por meio de interface programável,
viabilizam a formação de uma grande base de treinamento.
Para a realização deste trabalho foram coletados cerca de 50 milhões de mensagens filtradas apenas pelo idiomas inglês,
estes \textit{tweets} compõe a base de treinamento.
A base de teste por sua vez é composta de \textit{tweets} anotados manualmente. Essa base é composta da coletânea de dados
disponibilizados pela conferencia anual \textit{Semantic Evaluation} (SemEval)~\cite{semeval17} entre os anos de 2013 a 2017.
Esta base é composta de cerca de 70 mil \textit{tweets} e processo de anotação destes dados foi feito através da plataforma
CrowdFlower, mais detalhes sobre a formação da base de dados são apresentados por Rosenthal \textit{et al.}~\cite{rosenthal17},
organizadores do evento.

Primeiramente foram replicadas as técnicas abordadas por Go \textit{et al.}, \textit{Support Vector Machine} (SVM) e
\textit{Naïve Bayes} (NB), validando os resultados obtidos com o conjunto de dados coletados.
Em seguida, foi gerado um \textit{embedding}, pelo algoritmo Word2Vec, a partir da base de treinamento. Este treinamento
foi feito utilizando a biblioteca Gensim~\cite{gensim}.
Por fim, gerou-se um modelo de redes convolucionais aplicando como entrada a representação obtida pelo Word2Vec, como
descrito por Kim~\cite{kim14}.
A seleção do modelo foi determinada de maneira a maximizar a área sobre a curva de operação do receptor (ROC).
