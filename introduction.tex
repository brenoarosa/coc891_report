O processamento de linguagem natural (NLP), é um dos principais ramos da Inteligência Artificial.
Um de seus marcos inicias desse área de conhecimento foi o teste de Turing~\cite{turing50} no qual propõe a capacidade de
comunicação como um critério de inteligência.
Desde sua criação, no inicio dos anos 50, o NLP se consolidou de maneira a se fazer presente no dia-a-dia das pessoas como
pode-se observar na utilização de serviços automatizados de atendimento ao clientes, ferramentas de tradução etc.

Por sua vez, desde que Hinton \textit{et al.}~\cite{hinton06} apresentou o treinamento por camadas, \textit{Deep Learning}
passou a superar o estado da arte em diversas tarefas~\cite{lecun15} como reconhecimento de fala e detecção de objetos.
CNN~\cite{lecun98} inicialmente foram desenvolvidas para se aplicar em problemas de visão computacional.
Apesar da expansão do uso de redes \textit{multi-layer perceptron} (MLP) durante os anos 90, a aplicação de MLP em imagens
é computacionalmente custosa, visto que, normalmente, cada pixel se dispõe como uma dimensão da entrada.
Redes MLP apresentam outra dificuldade em problemas de visão computacional, por cada pixel ser considerado uma dimensão a
informação espacial contida na imagem é altamente sensível a deslocamentos, escalamentos e distorções.
LeCun \textit{et al.}~\cite{lecun98} projetaram as redes convolucionais de maneira a reduzir tais adversidades.

Assim como no processamento de imagens, o tratamento de texto também era dificultado pela dimensionalidade dos dados.
Isso se dá porque aplicações de NLP costumam trabalhar a nível de palavra, ou seja, um documento ou mensagem é substituído
por um vetor de palavras que por sua vez compõe a entrada do algoritmo de aprendizado de máquina a ser usado e tendo
cada palavra representada com codificação \textit{one-hot} de dicionários que são compostos de centenas de milhares de
palavras.
A esparsidade desse problema limitava a quantidade de técnicas aplicáveis, apenas com começo dos primeiros \textit{embeddings}
de palavras~\cite{bengio03}, dentre eles o \textit{Word2Vec}~\cite{mikolov13}, se tornou viável a utilização de técnicas
computacionalmente mais custosas.

Baseado nos êxitos obtidos tanto pelas redes convolutivas quanto pelos \textit{embeddings} de palavras, Yoon Kim~\cite{kim14}
apresenta um método de utilização de CNN para classificação de texto, comparando os resultados obtidos com relação as técnicas
tradicionalmente aplicadas a NLP.

Este artigo trata o trabalho de Kim~\cite{kim14} e aplicará as técnicas por ele apresentada na análise de sentimento de
microblogs com supervisão distante, como descrito por Go \textit{et at.}~\cite{go09}.
